% This work is licensed under the
%
%   Creative Commons Attribution-NonCommercial-ShareAlike 3.0 Unported
%
% license. Please see the file COPYING for more information.
\documentclass[10pt,landscape]{article}

\usepackage{alltt}
\usepackage{cprotect}
\usepackage{enumitem}
\usepackage[T1]{fontenc}
\usepackage[landscape,margin=13mm,footskip=1pt,includefoot]{geometry}
\usepackage{graphicx}
\usepackage{hyperref}
\usepackage[utf8]{inputenc}
\usepackage{multicol}
\usepackage{setspace}
%\usepackage[compact]{titlesec}
\usepackage{upquote}
\usepackage{xspace}

\newif\ifverbose

% The verbose flag controls the amount of content included. If \verbosetrue is
% defined, the cheat sheet will include numerous additional built-in functions
% that we deem too low-level for the majority of users. However, we still
% include their documentation in the source code for the sake of completeness,
% but you have to explicitly enable it by setting \verbosetrue (or simply
% uncommenting the line below).
%
%\verbosetrue

\graphicspath{{figs/}}

\pagestyle{empty}
\parindent=0pt

\setitemize{itemsep=1pt,topsep=0pt,parsep=1pt,leftmargin=10pt}

\hypersetup{
    colorlinks=true,        % false: boxed links; true: colored links
    linkcolor=red,          % color of internal links
    citecolor=cyan,         % color of links to bibliography
    filecolor=magenta,      % color of file links
    urlcolor=blue           % color of external links
}

\ifverbose
  \newcommand{\verbose}[1]{#1}
\else
  \newcommand{\verbose}[1]{}
\fi
\cMakeRobust{\verbose}

\newcommand{\todo}[1]{\textit{\textcolor{red}{TODO: #1}}}
\newcommand{\minisec}[1]{\textsc{#1}\\}
\newcommand{\first}{\emph{(i)}~}
\newcommand{\second}{\emph{(ii)}~}
\newcommand{\third}{\emph{(iii)}~}
\newcommand{\fourth}{\emph{(iv)}~}
\newcommand{\fifth}{\emph{(v)}~}

% Generic phrase when functions return a boolean flag that indicates whether
% the operation succeeded.
\newcommand{\ReturnsTrueOnSuccess}{Returns true on success.\xspace}

\begin{document}

\begin{multicols*}{3}

{\Huge\scshape
Bro\hspace{-2pt}\raisebox{15pt}{\tiny2.0}\hspace{-4pt}
Cheat Sheet}

\hfill\includegraphics[width=.4\linewidth]{bro-logo-small}
\vspace{-38pt}

{\scriptsize
\setstretch{1.5}
\begin{tabular}{l l}
Version: & \today\\
Website: & \url{http://www.bro-ids.org}\\
Email: & \texttt{\href{mailto:info@bro-ids.org}{info@bro-ids.org}}\\
Download: & \url{https://github.com/broids/cheat-sheet}\\
License: & \href{http://creativecommons.org/licenses/by-nc-sa/3.0/}
                {Attribution-NonCommercial-ShareAlike 3.0 Unported}
\end{tabular}
}

%\hfill
%\href{http://creativecommons.org/licenses/by-nc-sa/3.0/}
%{\includegraphics[width=.2\linewidth]{by-nc-sa}}
%\vspace{-10pt}

\vspace{-10pt}

\section*{Startup}

\texttt{bro \textit{[options] [file \dots]}}\\
\texttt{\textit{file}} \dotfill Bro policy script or \texttt{stdin}\\
\texttt{-e \textit{code}} \dotfill Augment policies by given code\\
\texttt{-h} \dotfill Display command line options\\
\texttt{-i \textit{iface}} \dotfill Read from given interface\\
\texttt{-p \textit{pfx}} \dotfill Add given prefix to policy resolution\\
\texttt{-r \textit{file}} \dotfill Read from given PCAP file\\
\texttt{-w \textit{file}} \dotfill Write to given file in PCAP format\\
\texttt{-x \textit{file}} \dotfill Print contents of state file\\
\texttt{-C} \dotfill Ignore invalid checksum

\section*{Language}

Lowercase letters represent instance variables and uppercase letters represent
types. In general, \texttt{x} is an instance of type \texttt{\textit{T}} and
\texttt{y} an instance of type \texttt{\textit{U}}. Argument names and record
fields begin with \texttt{a}, \texttt{b},~\ldots, and \texttt{z} represents a
default instance variable which takes on the type of the right-hand side
expression. For notational convenience, \texttt{x} can often be replaced with
an expression of type \texttt{\textit{T}}.

\subsection*{Variables}

Constant qualifier \dotfill \texttt{const}\\
Constant redefinition \dotfill \texttt{redef x \textit{op} \textit{expr}}\\
Scope qualifier \dotfill \texttt{local}, \texttt{global}\\
Declaration \dotfill \texttt{\textit{scope} x:~\textit{T}}\\
Declaration \& Definition \dotfill \texttt{\textit{scope} z = \textit{expr}}

\subsection*{Declarations}

Type \dotfill \texttt{type~\textit{name}:~\textit{T}}\\
Function \dotfill \texttt{function~f(a:~\textit{T},~...):~\textit{R}}\\
Event \dotfill \texttt{event~e(a:~\textit{T},~...)}

\subsection*{Modules}

Script import \dotfill \texttt{@load \textit{path}}\\
Set current namespace to \texttt{ns} \dotfill \texttt{module \textit{ns}}\\
Export global symbols \dotfill \verb|export { ... }|\\
Access \texttt{module} or \texttt{enum} namespace \dotfill \verb|T::a|

\subsection*{Statements}

Basic statement \dotfill
  \texttt{\textit{stmt};} or \texttt{\textit{expr};}\\
Code block \dotfill \texttt{\{ \textit{stmt}; ... \}}\\
Assignment \dotfill \texttt{z = \textit{expr}}\\
Function assignment \dotfill \texttt{z~=~function(...):~\textit{R}~\{..\}}\\
Event queuing \dotfill \verb|event e(...)|\\
Event scheduling \dotfill \verb|schedule 10 secs { e(...) }|\\
Print expression to \texttt{stdout} \dotfill \texttt{print \textit{expr}}\\

\vspace{-10pt}
%\minisec{Control Flow}
\begin{multicols*}{3}
\textsc{Branching}
\begin{alltt}
if (\textit{expr})
    \{ \ldots \}
else if (\textit{expr})
    \{ \ldots \}
else
    \{ \ldots \}
\end{alltt}

\textsc{Iteration}
\begin{alltt}
for (i in x)
    \{ \ldots \}
\end{alltt}

\textsc{Asynchronous}
\begin{alltt}
when (\textit{expr}) \{ \ldots \}
when (local x = \textit{expr}) \{ \ldots \}
\end{alltt}
\vspace{-10pt}

\begin{alltt}
\textsc{Control}\vspace{5pt}
break
continue
next
return
\end{alltt}
\end{multicols*}

\subsection*{Expressions}

\minisec{Operators}
\verb|!| \dotfill Negation\\
\verb|$|, \verb|?$| \dotfill Dereference, record field existence\\
\verb|+|, \verb|-|, \verb|*|, \verb|/|, \verb|%| \dotfill Arithmetic\\
\verb|++|, \verb|--| \dotfill Post-increment, post-decrement\\
\verb|+=|, \verb|-=|, \verb|*=|, \verb|/=| \dotfill Arithmetic and assignment\\
\verb|==|, \verb|!=| \dotfill Equality, inequality\\
\verb|<|, \verb|<=|, \verb|>=|, \verb|>|
  \dotfill Less/greater than (or equal)\\
\verb|&&|, \verb#||# \dotfill Conjunction, disjunction\\
\verb|in|, \verb|!in| \dotfill Membership or pattern matching\\
\verb|[x]| \dotfill Index strings and containers\\
\verb#|x|# \dotfill Cardinality/size for strings and containers\\
\verb|f(...)| \dotfill Function call\\
\texttt{\textit{expr}$\;$?$\;$\textit{expr}$\;$:$\;$\textit{expr}}
  \dotfill Ternary if-then-else

\subsection*{Types}

\minisec{Basic}
\verb|addr| \dotfill IP address (\verb|127.0.0.1|)\\
\verb|bool| \dotfill Boolean flag (\verb|T|, \verb|F|)\\
\verb|count| \dotfill 64-bit unsigned integer (\verb|42|)\\
\verb|double| \dotfill Double-precision floating point (\verb|99.9|)\\
\verb|int| \dotfill 64-bit signed integer (\verb|-7|)\\
\verb|interval| \dotfill Time interval
  (\verb|8 sec|/\verb|min|/\verb|hr|/\verb|day[s]|)\\
\verb|pattern| \dotfill Regular expression (\verb|/^br[oO])$/|)\\
\verb|port| \dotfill Transport-layer port
  (\verb|22/tcp|, \verb|53/udp|)\\
\verb|string| \dotfill String of bytes (\verb|"foo"|)\\
\verb|subnet| \dotfill CIDR subnet mask (\verb|10.0.0.0/8|)\\
\verb|time| \dotfill Absolute epoch time (\verb|1320977325|)\\

\minisec{Enumerables}
Declaration \dotfill \verb|enum { FOO, BAR }|\\
Assignment \dotfill \texttt{\textit{scope}} \verb|x = FOO|\\

\minisec{Records}
Declaration \dotfill \texttt{record~\{~a:~\textit{T},~b:~\textit{U},~... \}}\\
Constructor \dotfill \verb|record($a=x, $b=y, ...)|\\
Assignment \dotfill \texttt{\textit{scope}} \verb|r = [$a=x, $b=y, ...]|\\
Access \dotfill \verb|z = r$a|\\
Field assignment \dotfill \verb|r$b = y|\\
Deletion \dotfill \verb|delete r$a|\\

\minisec{Sets}
Declaration \dotfill \texttt{set[\textit{T}]}\\
Constructor \dotfill \verb|set(x, ...)|\\
Assignment \dotfill \texttt{\textit{scope}} \verb|s = { x, ... }|\\
Access \dotfill \verb|z = s[x]|\\
Insertion \dotfill \verb|add s[x]|\\
Deletion \dotfill \verb|delete s[x]|\\

\minisec{Tables}
Declaration \dotfill \texttt{table[\textit{T}]~of~\textit{U}}\\
Constructor \dotfill \verb|table([x] = y, ...)|\\
Assignment \dotfill \texttt{\textit{scope}} \verb|t = { [x] = y, ... }|\\
Access \dotfill \verb|z = t[x]|\\
Insertion \dotfill \verb|t[x] = y|\\
Deletion \dotfill \verb|delete t[x]|\\

\minisec{Vectors}
Declaration \dotfill \texttt{vector~of~\textit{T}}\\
Constructor \dotfill \verb|vector(x, ...)|\\
Assignment \dotfill \texttt{\textit{scope}} \verb|v = { x, ... }|\\
Access \dotfill \verb|z = v[0]|\\
Insertion \dotfill \verb|v[42] = x|
% FIXME: not yet supported in Bro, see #679.
%Deletion \dotfill \verb|delete v[3]|
\end{multicols*}

\begin{multicols*}{2}
\subsection*{Attributes}
Attributes occur at the end of type/event declarations and change their
behavior. The syntax is \verb|&key| or \verb|&key=val|, e.g.,
\verb|type T: set[count] &read_expire=5min| or
\verb|event foo() &priority=-3|.\\

\verb|&optional| \dotfill Allow record field to be missing\\
\verb|&default=x| \dotfill Use default value \texttt{x} for record fields and
  container elements\\
\verb|&redef| \dotfill Allow for redefinition of initial object value\\
% Users should use the logging framework instead.
%\verb|&rotate_interval=x| \dotfill Rotate file after time \verb|x|\\
%\verb|&rotate_size=x| \dotfill Rotate file after reaching size limit \verb|x|\\
%
% These do not seem to be worthwhile including
%\verb|&add_func=f| \dotfill Call \verb|f| after adding element to container\\
%\verb|&delete_func=f| \dotfill Call \verb|f| before deleting element from
%container \\
\verb|&expire_func=f| \dotfill Call \verb|f| right before container element
  expires\\
\verb|&read_expire=x| \dotfill Remove element after not reading it for time
  \verb|x|\\
\verb|&write_expire=x| \dotfill Remove element after not writing it for time
  \verb|x|\\
\verb|&create_expire=x| \dotfill Remove element after time \verb|x| from
  insertion\\
\verb|&persistent| \dotfill Write state to disk (per default on shutdown)\\
\verb|&synchronized| \dotfill Synchronize variable across nodes\\
\verb|&raw_output| \dotfill Do not escape non-ASCII characters when writing to
  a file\\
\verb|&mergeable| \dotfill Prefer set union to assignment for synchronized
  state\\
\verb|&priority=x| \dotfill Execution priority of event handler, higher values
first, default 0\\
  %$\mathtt{x} \in [-10,10]$ by convention\\
\verb|&group="x"| \dotfill Events in the same group can be jointly
  activated/deactivated\\
\verb|&log| \dotfill Write record field to log \\

\subsection*{Built-In Functions (BIFs)}
\linespread{0.9}

\subsubsection*{Core}

\begin{itemize}
\verbose{
  \item \verb|getenv(var: string): string|\\
    Returns the system environment variable identified by \verb|var|, or an
    empty string if it is not defined.
  \item \verb|setenv(var: string, val: string): bool|\\
    Sets the system environment variable \verb|var| to \verb|val|.
    \ReturnsTrueOnSuccess
}
  \item \verb|syslog(s: string)|\\
    Send the string \verb|s| to syslog.
  \item \verb|system(s: string): int|\\
    Invokes a command via the \texttt{system} function.
    Returns the return value from the \texttt{system()} call.
\verbose{
    Note that this corresponds to the status of backgrounding the given
    command, not to the exit status of the command itself. A value of 127
    corresponds to a failure to execute \verb|sh|, and -1 to an internal system
    failure.
}
    The command is run in the background, \verb|stdout| redirects to
    \verb|stderr|. Here is a usage example:
    \verb|system(fmt("rm \"%s\"", str_shell_escape(sniffed_data)));|
\verbose{
  \item \verb|system_env(s: string, env: any): int|\\
    Same as \verb|system|, but prepare the environment before invoking the
    command \verb|s| with the set/table \verb|env|.
}
  \item \verb|piped_exec(program: string, to_write: string): bool|\\
    Opens the application \verb|program| with \verb|popen| and writes the
    string \verb|to_write| to \texttt{stdin} of the opened program.
  \item \verb|srand(seed: count)|\\
    Sets the seed for subsequent \verb|rand| calls.
  \item \verb|rand(max: count): count|\\
    Returns a random value from the interval $[0, \mathtt{max})$.
  \item \verb|md5_hash(...): string|\\
    Computes the MD5 hash value of the provided list of arguments.
  \item \verb|md5_hash_init(index: any): bool|\\
    Initializes MD5 state for \verb|index| to allow for computing hash values
    incrementally via the function \verb|md5_hash_update|.
\verbose{
    For example, when computing incremental MD5 values of transferred files in
    multiple concurrent HTTP connections, it is necessary to call
    \verb|md5_hash_init(c$id)| once before invoking
    \verb|md5_hash_update(c$id, some_more_data)| in the \verb|http_entity_data|
    event handler.
}
  \item \verb|md5_hash_update(index: any, data: string): bool|\\
    Updates the MD5 value associated with \verb|index|. Note that it is
    necessary to call \verb|md5_hash_init(index)| once before calling this
    function to initialize the MD5 state.
  \item \verb|md5_hash_finish(index: any): string|\\
    Returns the final MD5 digest associated with the internal state identified
    by \verb|index|.
  \item \verb|md5_hmac(...): string|\\
    Computes an HMAC-MD5 hash value of the provided list of arguments. The HMAC
    secret key is generated from available entropy when Bro starts up, or it
    can be specified for repeatability using the \texttt{-K} flag.
  \item \verb|strftime(fmt: string, d: time): string|\\
    Formats the time value \verb|d| according to the format string \verb|fmt|.
    See \verb|man strftime| for the format of \verb|fmt|.
  \item \verb|lookup_addr(host: addr): string|\\
    Issues an asynchronous reverse DNS lookup and delays the function result.
    Therefore, it can only be called inside a \verb|when|-condition, e.g.,
    \verb|when ( local host = lookup_addr(10.0.0.1) ) { f(host); }|.
    Returns the DNS name of \verb|host|.
  \item \verb|lookup_hostname(host: string): set[addr]|\\
    Issues an asynchronous DNS lookup and delays the function result.
    Returns a set containing the addresses that \verb|host| resolves to.
    See \verb|lookup_addr| for a usage example.
  \item \verb|identify_data(data: string, return_mime: bool): string|\\
    Invokes \texttt{libmagic} on \verb|data| to determine its MIME type. If
    \verb|return_mime| is true, the function returns a MIME type string instead
    of a textual description.
  \item \verb|unique_id(prefix: string): string|\\
    Creates an identifier that is unique with high probability, with
    \verb|prefix| prepended to the result.
  \item \verb|unique_id_from(pool: int, prefix: string): string|\\
    Same as \verb|unique_id|, except that the additional argument \verb|pool|
    specifies a seed for determinism.
  \item \verb|terminate(): bool|
    Gracefully shuts down Bro by terminating outstanding processing. Returns
    true after successful termination and false when Bro is still in the
    process of shutting down.
  \item \verb|exit(code: int)|
    Shuts down the Bro process immediately and returns with \verb|code|.
\end{itemize}

% Users typically do not interact with the packet filtering directly.
\verbose{
\subsubsection*{Packet Filtering}

\begin{itemize}
  \item \verb|precompile_pcap_filter(id: PcapFilterID, s: string): bool|\\
    Precompiles the PCAP filter \verb|s| and binds it to the identifier
    \verb|id| in \texttt{libpcap}. Returns true if the filter expression is
    valid. See \verb|install_pcap_filter|.
  \item \verb|install_pcap_filter(id: PcapFilterID): bool|\\
    Installs a PCAP filter precompiled via \verb|precompile_pcap_filter|.
    Returns true if the installation succeeds.
  \item \verb|install_src_addr_filter(ip: addr, flags: count, p double): bool|\\
    Installs a filter to drop packets from the IP source address \verb|ip| with
    probability $\mathtt{p} \in [0,1]$ if none of the TCP flags given by
    \verb|flags| are set.
  \item \verb|install_src_net_filter(s: subnet, flags: count, p: double): bool|\\
    Same as \verb|install_src_addr_filter| but for subnets instead of IP
    addresses.
  \item \verb|uninstall_src_addr_filter(ip: addr): bool|\\
    Removes an IP source address filter for \verb|ip| installed with
    \verb|install_src_addr_filter|.
  \item \verb|uninstall_src_net_filter(snet: subnet): bool|\\
    Removes an IP source subnet filter for \verb|snet| installed with
    \verb|install_src_net_filter|.
  \item \verb|install_dst_addr_filter(ip: addr, flags: count, p: double): bool|\\
    Same as \verb|install_src_addr_filter| but for IP destination addresses.
  \item \verb|install_dst_net_filter(s: snet, flags: count, p: double): bool|\\
    Same as \verb|install_dst_addr_filter| but for subnets instead of IP
    addresses.
  \item \verb|uninstall_dst_addr_filter(ip: addr): bool|\\
    Removes an IP destination address filter for \verb|ip| installed with
    \verb|install_dst_addr_filter|.
  \item \verb|uninstall_dst_net_filter(snet: subnet): bool|\\
    Removes an IP destination subnet filter for \verb|snet| installed with
    \verb|install_dst_net_filter|.
  \item \verb|pcap_error(): string|\\
    Returns a descriptive error message if the last PCAP function failed.
\end{itemize}
}

\subsubsection*{Introspection}

\begin{itemize}
  \item \verb|bro_version(): string|\\
    Returns the Bro version string.
  \item \verb|getpid(): count|\\
    Returns Bro's process ID.
  \item \verb|gethostname(): string|\\
    Returns the hostname of the machine Bro runs on.
  \item \verb|current_time(): time|\\
    Returns the current wall-clock time.
\verbose{
    In general, you should use \verb|network_time| instead of
    \verb|current_time| unless you're using Bro for non-networking uses (such
    as general scripting; not particularly recommended), because otherwise your
    script may behave very differently on live traffic versus played-back
    traffic from a save file.
}
  \item \verb|network_time(): time|\\
    Returns the timestamp of the last packet processed.
    Returns the timestamp of the most recently read packet, whether read from a
    live network interface or from a save file.
\verbose{
  \item \verb|do_profiling()|\\
    Enables detailed collections of statistics about CPU/memory usage,
    connections, TCP states/reassembler, DNS lookups, timers, and script-level
    state. The script variable \verb|profiling_file| holds the name of the log
    file.
  \item \verb|net_stats(): NetStats|\\
    Returns statistics about the number of packets
    \first received by Bro,
    \second dropped,
    and \third seen on the link (not always available).
  \item \verb|resource_usage(): bro_resources|\\
    Returns Bro process statistics, such as real/user/sys CPU time, memory
    usage, page faults, number of TCP/UDP/ICMP connections, timers, and events
    queued/dispatched.
  \item \verb|get_matcher_stats(): matcher_stats|\\
    Returns statistics about the regular expression engine, such as the number
    of distinct matchers, DFA states, DFA state transitions, memory usage of
    DFA states, cache hits/misses, and average number of NFA states across all
    matchers.
  \item \verb|dump_rule_stats(f: file): bool|\\
    Write rule matcher statistics (DFA states, transitions, memory usage, cache
    hits/misses) to the file \verb|f|.
    \ReturnsTrueOnSuccess
  \item \verb|get_gap_summary(): gap_info|\\
    Returns statistics about TCP gaps.
  \item \verb|type_name(t: any): string|\\
    Returns the type name of \texttt{t}.
  \item \verb|record_type_to_vector(rt: string): vector of string|\\
    Converts the record type name \texttt{rt} into a vector of strings, where
    each element is the name of a record field. Nested records are flattened.
  \item \verb|global_sizes(): table[string] of count|\\
    Returns a table containing the size of all global variables, where the
    index is the variable name and the value the variable size in bytes.
  \item \verb|global_ids(): table[string] of script_id|\\
    Returns a table with information about all global identifiers. The table
    value is a record containing the type name of the identifier, whether it is
    exported, a constant, an enum constant, redefinable, and its value (if it
    has one).
  \item \verb|lookup_ID(id: string): any|\\
    Returns the value associated with the global identifier \verb|id|. If
    \verb|id| does not describe a valid identifier, the function returns the
    string \verb|"<unknown id>"| or \verb|"<no ID value>"|.
  \item \verb|record_fields(r: any): table[string] of record_field|\\
    Returns meta data about a record instance \verb|r|, which includes the
    type name, whether the field is logged, its value (if it has one), and its
    default value (if specified).
  \item \verb|reading_live_traffic(): bool|\\
    Checks whether Bro reads traffic from one or more network interfaces (as
    opposed to from a network trace in a file). Note that this function returns
    true even after Bro has stopped reading network traffic, for example due to
    receiving a termination signal.
  \item \verb|reading_traces(): bool|\\
    Checks whether Bro reads traffic from a trace file (as opposed to
    from a network interface).
  \item \verb|bro_is_terminating(): bool|\\
    Returns true if Bro is in the process of shutting down.
}
  \item \verb|is_local_interface(ip: addr): bool|\\
    Returns true if the address \verb|ip| is a valid DNS entry for
    \texttt{localhost}.
\verbose{
  \item \verb|is_external_connection(c: connection): bool|\\
    Returns true if the connection \verb|c| has been received externally.
    Broccoli or the Time Machine can send packets to Bro via a mechanism that
    one step lower than sending events. This function returns true if the
    \verb|c| stems from one of these other \emph{packet sources}.
  \item \verb|disable_print_hook(f: file)|\\
    Function equivalent of the \verb|&disable_print_hook| attribute. In a
    distributed setup, communicating Bro instances generate the event
    \verb|print_hook| for each print statement and send it to the remote side.
    When disabled for a particular file, these events will not be propagated to
    other peers.
  \item \verb|enable_communication()|\\
    Enables the communication system. By default, communication is off until
    explicitly enabled and all other calls to communication-related BiF's will
    be ignored until done so.
  \item \verb|suspend_processing()|\\
    Stops Bro's packet processing. Used to synchronize distributed trace
    processing with communication (\emph{pseudo-realtime} mode).
  \item \verb|continue_processing()|\\
    Resumes Bro's packet processing; the counterpart to
    \verb|suspend_processing|.
  \item \verb|suspend_state_updates()|\\
    Stops propagating \verb|&synchronized| accesses.
  \item \verb|resume_state_updates()|\\
    Resumes propagating \verb|&synchronized| accesses; the counterpart to
    \verb|suspend_state_udpates|.
  \item \verb|enable_event_group(group: string)|\\
    Enables all event handlers in the group \verb|group|. This affects all
    handlers that have been tagged with the attribute \verb|&group="group"|.
  \item \verb|disable_event_group(group: string)|\\
    Disables all event handlers in the group \verb|group|. This affects all
    handlers that have been tagged with the attribute \verb|&group="group"|.
}
\end{itemize}

\verbose{
\subsubsection*{Independent State}

\begin{itemize}
  \item \verb|checkpoint_state(): bool|\\
    Flushes in-memory state with the \verb|&persistence| attribute to the state
    file \texttt{.state/state.bst}.
% FIXME: deprecated, will be removed
%  \item \verb|dump_config(): bool|\\
%    Flushes all global identifiers into the file \texttt{.state/config.bst}.
  \item \verb|rescan_state(): bool|\\
    Reads persistent configuration and state from the \texttt{.state}
    directory.
  \item \verb|capture_events(filename: string): bool|\\
    Writes the event stream generated by the core to \verb|filename|. Use the
    \texttt{-x} command line switch to replay the saved events.
  \item \verb|capture_state_updates(filename: string): bool|\\
    Writes state updates generated by \verb|&synchronized| variables to the
    file \verb|filename|.
  \item \begin{verbatim}
connect(ip: addr, p: port, our_class: string, retry: interval,
        ssl: bool): count
\end{verbatim}
    Establishes a connection to a remote Bro instance or Broccoli application
    at IP address \verb|ip| and port \verb|p|. If the connection fails, Bro
    tries to reconnect with the peer after the time interval \verb|retry|. If
    \verb|ssl| is true, the connection uses SSL to encrypt the session.
    If \verb|our_class| is a non-empty string, the remote (listening) peer
    checks it against its class name in its peer table and terminates the
    connection if they don't match.
    Returns a locally unique ID of the new peer.
  \item \verb|disconnect(p: event_peer): bool|\\
    Disconnects the peer identified by \verb|p|.
  \item \verb|listen(ip: addr, p: port, ssl: bool): bool|\\
    Listens on address \verb|ip| and port \verb|p| for remote connections. If
    \verb|ssl| is true, Bro uses SSL to encrypt the session.
    \ReturnsTrueOnSuccess
  \item \verb|request_remote_events(p: event_peer, handlers: pattern): bool|\\
    Subscribes to all events from remote peer \verb|p| whose names match the
    pattern \verb|handlers|.
    %That is, the local Bro instance inserts all events received from \verb|p|
    %into its own event queue and dispatches them when they are ready.
  \item \verb|request_remote_sync(p: event_peer, auth: bool): bool|\\
    Requests synchronization of IDs with remote peer \verb|p|. If \verb|auth|
    is true, the local Bro instance considers its current state authoritative
    and sends it to \verb|p| right after the handshake.
  \item \verb|request_remote_logs(p: event_peer): bool|\\
    Requests logs from remote peer \verb|p|.
    \ReturnsTrueOnSuccess
  \item \verb|set_accept_state(p: event_peer, accept: bool): bool|\\
    Sets a boolean flag whether Bro accepts state from the remote peer \verb|p|.
    \ReturnsTrueOnSuccess
  \item \verb|set_compression_level(p: event_peer, level: count): bool|\\
    Sets the compression level of the session with the remote peer \verb|p|.
    values for \verb|level| are in $[0,9]$, where 0 is the default and means no
    compression.
    \ReturnsTrueOnSuccess
  \item \verb|is_remote_event(): bool|\\
    Returns true if the last raised event came from a remote peer.
  \item \verb|send_state(p: event_peer): bool|\\
    Sends all persistent state to the remote peer \verb|p|.
    \ReturnsTrueOnSuccess
  \item \verb|send_id(p: event_peer, id: string): bool|\\
    Sends the global identifier \verb|id| to the remote peer \verb|p|, which
    might then install it locally.
  \item \verb|terminate_communication(): bool|\\
    Gracefully finishes communication by first making sure that all remaining
    data from parent and child has been sent out.
    Returns true if the termination process has been started successfully.
  \item \verb|complete_handshake(p: event_peer): bool|\\
    Signals the remote peer \verb|p| that the local Bro instance finished the
    initial handshake.
    \ReturnsTrueOnSuccess
  \item \verb|send_ping(p: event_peer, seq: count): bool|\\
    Sends a ping with a sequence number \verb|seq| to the remote peer \verb|p|.
    In combination with an event handler for \verb|remote_pong|, this function
    can be used to measure latency between two peers.
    \ReturnsTrueOnSuccess
  \item \verb|send_current_packet(p: event_peer): bool|\\
    Sends the currently processed packet to the remote peer \verb|p|.
    \ReturnsTrueOnSuccess
  \item \verb|get_event_peer(): event_peer|\\
    Returns the peer who generated the last event.
  \item \verb|get_local_event_peer(): event_peer|\\
    Returns the local peer.
  \item \verb|send_capture_filter(p: event_peer, s: string): bool|\\
    Sends the capture filter \verb|s| to the remote peer \verb|p|.
    \ReturnsTrueOnSuccess
% FIXME: deprecated, will be removed
%  \item \verb|make_connection_persistent(c: connection)|\\
%    Makes the connection \verb|c| persistent.
\end{itemize}
}

\subsubsection*{Analyzer Behavior}

\begin{itemize}
\verbose{
  \item \verb|current_analyzer(): count|\\
    Returns the ID of the analyzer which raised the current event, or 0 if no
    analyzer has been instantiated.
  \item \verb|analyzer_name(aid: count): string|\\
    Translates the analyzer ID \verb|aid| to a string representation.
  \item
\begin{verbatim}
expect_connection(orig: addr, resp: addr, resp_p: port,
          				analyzer: count, tout: interval)
\end{verbatim}
    Schedules the analyzer identified by the ID \verb|analyzer| for a future
    connection from IP address \verb|orig| to \verb|resp| at port
    \verb|resp_p|. The function ignores the scheduling request if the
    connection did not occur within the specified time interval \verb|tout|.
  \item \verb|disable_analyzer(id: conn_id, aid: count): bool|\\
    Disables the analyzer \verb|aid| which raised the current event if it
    belongs to connection identified by \verb|id|.
  \item \verb|get_login_state(id: conn_id): count|\\
    Returns the state of the given login (Telnet or Rlogin) connection
    identified by \texttt{id}. Returns false if the connection is not active
    or is not tagged as a login analyzer. Otherwise the function returns the
    state, which can be one of:
    \begin{itemize}
      \item \verb|LOGIN_STATE_AUTHENTICATE|: The connection is in its initial
        authentication dialog.
      \item \verb|OGIN_STATE_LOGGED_IN|: The analyzer believes the user has
        successfully authenticated.
      \item \verb|LOGIN_STATE_SKIP|: The analyzer has skipped any further
        processing of the connection.
      \item \verb|LOGIN_STATE_CONFUSED|: The analyzer has concluded that it
        does not correctly know the state of the connection, and/or the
        username associated with it.
    \end{itemize}
  \item \verb|set_login_state(id: conn_id, new_state: count): bool|\\
    Sets the login state of the connection identified by \verb|id| to
    \verb|new_state|. See \verb|get_login_state| for potential values of
    \verb|new_state|. Returns false if \texttt{id} is not an active connection
    or does not tagged as login analyzer, and true otherwise.
}
  \item \verb|skip_further_processing(id: conn_id): bool|\\
    Informs Bro that it should skip any further processing of the contents of
    the connection identified by \verb|id|. In particular, Bro will refrain
    from reassembling the TCP byte stream and from generating events relating
    to any analyzers that have been processing the connection.
\verbose{
    Bro will still generate connection-oriented events such as
    \verb|connection_finished|.
}
    Returns false if \verb|id| does not point to an active connection and
    true otherwise.
\verbose{
    Note that this does not in itself imply that packets from
    this connection will not be recorded, which is controlled separately by
    \verb|set_record_packets|.
}
  \item \verb|set_record_packets(id: conn_id, do_record: bool): bool|\\
    Controls whether packet contents belonging to the connection identified by
    \verb|id| should be recorded (when \texttt{-w \textit{out.pcap}} is
    provided on the command line).
    Note that this is independent of whether Bro processes the packets of this
    connection, which is controlled separately by
    \verb|skip_further_processing|.
  \item \verb|set_contents_file(id: conn_id, direction: count, f: file): bool|\\
    Associates the file handle \verb|f| with the connection identified by
    \verb|id| for writing TCP byte stream contents. The argument
    \verb|direction| can take one the four values
    \verb|CONTENTS_{NONE,ORIG,RESP,BOTH}| and controls what sides of the
    connection contents are recorded.
\verbose{
    \begin{itemize}
      \item \verb|CONTENTS_NONE|: Stop recording the connection's content.
      \item \verb|CONTENTS_ORIG|: Record the data sent by the connection
        originator (often the client).
      \item \verb|CONTENTS_RESP|: Record the data sent by the connection
        responder (often the server).
      \item \verb|CONTENTS_BOTH|: Record the data sent in both directions.
        Results in the two directions being intermixed in the file, in the
        order the data was seen by Bro.
    \end{itemize}
}
    Returns false if \verb|id| does not point to an active connection and
    true otherwise.
\verbose{
    Note that the data recorded to the file reflects the byte
    stream, not the contents of individual packets. Reordering and duplicates
    are removed. If any data is missing, the recording stops at the missing
    data; this can happen, e.g., due to an \verb|ack_above_hole| event.
}
  \item \verb|get_contents_file(id: conn_id, direction: count): file|\\
    Returns the file handle associated with the connection identified by
    \texttt{id} and \texttt{direction}. If the connection exists but no
    contents file for \texttt{direction}, the function returns a handle to new
    file. If not active connection for \texttt{id} exists, it returns an error.
  \item \verb|skip_http_entity_data(c: connection, is_orig: bool)|\\
    Skips the data of the HTTP entity in the connection \texttt{c}. If
    \verb|is_orig| is true, the client data is skipped and the server data
    otherwise.
  \item \verb|skip_smtp_data(c: connection)|\\
    Skips SMTP data until the next email in \texttt{c}.
  \item \verb|dump_current_packet(file_name: string): bool|\\
    Writes the current packet to the file identified by \verb|file_name|.
    \ReturnsTrueOnSuccess
\verbose{
  \item \verb|get_current_packet(): pcap_packet|\\
    Returns the currently processed PCAP packet, which is a record containing
    the timestamp, ``snaplen,'' and packet data.
  \item \verb|dump_packet(pkt: pcap_packet, file_name: string): bool|\\
    Writes the packet \verb|pkt| to the file identified by \verb|file_name|.
    \ReturnsTrueOnSuccess
  \item \verb|set_inactivity_timeout(id: conn_id, t: interval): interval|\\
    Sets an individual inactivity timeout for the connection identified by
    \texttt{id} (overrides the global inactivity timeout).
    Returns the previous timeout interval.
}
\end{itemize}

\subsubsection*{Files and Directories}

\begin{itemize}
  \item \verb|open(f: string): file|\\
    Opens the file identified by \texttt{f} for writing. Returns a handle
    for subsequent file operations.
  \item \verb|open_for_append(f: string): file|\\
    Same as \texttt{open}, except that \texttt{f} is not overwritten and
    content is appended at the end of the file.
  \item \verb|close(f: file): bool|\\
    Closes the file handle \texttt{f} and flushes buffered content. Returns
    true on success.
  \item \verb|active_file(f: file): bool|\\
    Checks whether \texttt{f} is open.
  \item \verb|write_file(f: file, data: string): bool|\\
    Writes \texttt{data} to \texttt{f}. Returns true on success.
  \item \verb|file_size(f: string): double|\\
    Returns the file size in bytes of the file identified by \verb|f|.
  \item \verb|get_file_name(f: file): string|\\
    Returns the filename associated with \texttt{f}.
  \item \verb|set_buf(f: file, buffered: bool)|\\
    Alters the buffering behavior of \texttt{f}. When \texttt{buffered} is
    true, the file is fully buffered, i.e., bytes are saved in a buffered until
    the block size has been reached. When \texttt{buffered} is false, the file
    is line buffered, i.e., bytes are saved up until a newline occurs.
  \item \verb|flush_all(): bool|\\
    Flushes all open files to disk.
    Returns true when the operations(s) succeeded.
  \item \verb|mkdir(f: string): bool|\\
    Creates a new directory identified by \verb|f|. Returns true if the
    operation succeeds and false if the creation fails or if \verb|f| exists
    already.
  \item \verb|enable_raw_output(f: file)|\\
    Function equivalent to the \verb|&raw_output| attribute, which prevents
    escaping of non-ASCII characters when writing to \verb|f|.
\verbose{
  \item \verb|rotate_file(f: file)|\\
    Rotates the file \verb|f|. Returns rotation statistics which include the
    original file name, the name after the rotation, and the time when \verb|f|
    was opened/closed.
  \item \verb|rotate_file_by_name(f: string)|\\
    Same as \verb|rotate_file|, but uses the filename rather than the handle to
    identify the file.
  \item \verb|calc_next_rotate(i: interval): interval|\\
    Calculates the duration until the next time a file is to be rotated based
    on the given rotate interval \verb|i|.
}
\end{itemize}

\subsubsection*{Generic Programming}

\begin{itemize}
\verbose{
  \item \verb|same_object(o1: any, o2: any): bool|\\
    Checks whether \texttt{o1} and \texttt{o2} reference the same internal
    object.
  \item \verb|val_size(v: any): count|\\
    Returns the number bytes that \verb|v| occupies in memory.
}
  \item \verb|length(v: any): count|\\
    Returns the number of elements in the container \texttt{v}.
  \item \verb|clear_table(v: any)|\\
    Removes all elements from the set or table \texttt{v}.
  \item \verb|resize(v: any, newsize: count): count|.
    Resizes the vector \verb|v| to the size \verb|newsize|.
    Returns the old size of \verb|v| and 0 if \verb|v| is not a \verb|vector|
    type.
  \item \verb|any_set(v: any): bool|\\
    Tests whether the boolean vector (\verb|vector of bool|) has any true
    element, i.e., checks whether $\exists x \in \mathtt{v}: x = \mathtt{T}$.
  \item \verb|all_set(v: any): bool|\\
    Tests whether all elements of the boolean vector (\verb|vector of bool|) are
    true, i.e., checks whether $\forall x \in \mathtt{v}: x = \mathtt{T}$.
    Missing elements count as false.
  \item \verb|sort(v: any, ...): any|\\
    Sorts the vector \verb|v| in place and returns the original vector.
    The second argument is a comparison function that takes two arguments: if
    the type of \verb|v| is \verb|vector of T|, then the comparison function
    must be \verb|function(a: T, b: T): bool|, which returns \verb|a < b| for
    some type-specific notion of the less-than operator.
  \item \verb|order(v: any, ...): vector of count|\\
    Returns the order of the elements in the vector \verb|v| according to some
    comparison function. See \verb|sort|.
\end{itemize}

\subsubsection*{Math}

\begin{itemize}
  \item \verb|floor(x: double): double|\\
    Chops off any decimal digits of \texttt{x},
    i.e., computes $\lfloor\mathtt{x}\rfloor$.
  \item \verb|sqrt(x: double): double|\\
    Returns the square root of \texttt{x}, i.e., computes $\sqrt{\mathtt{x}}$.
  \item \verb|exp(x: double): double|\\
    Raises $e$ to the power of \texttt{x}, i.e., computes $e^\mathtt{x}$.
  \item \verb|ln(x: double): double|\\
    Returns the natural logarithm of \texttt{x},
    i.e., computes $\ln \mathtt{x}$.
  \item \verb|log10(x: double): double|\\
    Returns the common logarithm of \texttt{x},
    i.e., computes $\log_{10} \mathtt{x}$.
\end{itemize}

\subsubsection*{String Processing}

\begin{itemize}
  \item \verb|byte_len(s: string): count|\\
    Returns the number of characters (i.e., bytes) in \texttt{s}.  This
    includes any embedded NULs, and also a trailing NUL, if any (which is why
    the function isn't called \verb|strlen|; to remind the user that Bro
    strings can include NULs).
  \item \verb|sub_bytes(s: string, start: count, n: int): string|\\
    Extracts a substring of \texttt{s}, starting at position \texttt{start} and
    having length \texttt{n}.
  \item \verb|split(s: string, re: pattern): table[count] of string|\\
    Splits \texttt{s} into an array using \texttt{re} to separate the elements.
    The returned table starts at index 1. Note that conceptually the return
    value is meant to be a vector and this might change in the future.
  \item \verb|split1(s: string, re: pattern): table[count] of string|\\
    Same as \texttt{split}, but \texttt{s} is only split once (if possible) at
    the earliest position and an array of two strings is returned. An array of
    one string is returned when \texttt{s} cannot be split.
  \item \verb|split_all(s: string, re: pattern): table[count] of string|\\
    Same as \texttt{split}, but also include the matching separators, e.g.,
    \verb|split_all("a-b--cd", /(\-)+/)| returns
    \verb|{"a", "-", "b", "--", "cd"}|. Odd-indexed elements do not match the
    pattern and even-indexed ones do.
  \item
\begin{verbatim}
split_n(s: string, re: pattern, incl_sep: bool,
        max_num_sep: count): table[count] of string
\end{verbatim}
    Similar to \verb|split1| and \verb|split_all|, but \verb|incl_sep|
    indicates whether to include matching separators and \verb|max_num_sep| the
    number of times to split \texttt{s}.
  \item \verb|str_split(s: string, idx: vector of count): vector of string|\\
    Splits \texttt{s} into substrings, taking all the indices in
    \texttt{idx} as cutting points; \texttt{idx} does not need to be sorted and
    out-of-bounds indices are ignored.
  \item \verb|string_cat(...): string|\\
    Concatenes a variable number of string arguments into a single string.
  \item \verb|cat_string_array(a: table[count] of string): string|\\
    Same as \verb|string_cat|, except that it takes an array of strings as
    argument and concatenates its values into a single string.
  \item
\begin{verbatim}
cat_string_array_n(a: table[count] of string,
                   start: count, end: count): string
\end{verbatim}
    Same as \verb|cat_string_array|, but only concatenates the strings from
    index \verb|start| to \verb|end|.
  \item
\begin{verbatim}
join_string_array(sep: string, a: table[count] of string): string
\end{verbatim}
    Concatenates all elements in \verb|a| into a single string,
    with \verb|sep| placed between each element.
  \item \verb|join_string_vec(v: vector of string, sep: string): string|\\
    Concatenates all elements in \verb|v| into a single string,
    with \verb|sep| placed between each element.
  \item \verb|sort_string_array(a: table[count] of string): string|\\
    Sorts the string array \verb|a| and returns a sorted copy.
  \item \verb|sub(s: string, re: pattern, repl: string): string|\\
    Substitutes \texttt{repl} for the first occurrence of \texttt{re} in
    \texttt{s}.
  \item \verb|gsub(s: string, re: pattern, repl: string): string|\\
    Same as \texttt{sub} except that \emph{all} occurrences of \texttt{re} are
    replaced.
  \item \verb|strcmp(s1: string, s2: string): int|
    Lexicographically compares \texttt{s1} and \texttt{s2}. Returns an integer
    greater than, equal to, or less than 0 according as \texttt{s1} is greater
    than, equal to, or less than \texttt{s2}.
  \item \verb|strstr(big: string, little: string): count|\\
    Locates the first occurrence of \texttt{little} in \texttt{big}.
    Returns 0 if \texttt{little} is not found in \texttt{big}.
  \item \verb|subst_string(s: string, from: string, to: string): string|\\
    Substitutes each (non-overlapping) appearance of \texttt{from} in
    \texttt{s} to \texttt{to}, and return the resulting string.
  \item \verb|to_lower(s: string): string|\\
    Returns a copy of the given string with the uppercase letters (as indicated
    by \verb|isascii| and \verb|isupper|) folded to lowercase (via
    \verb|tolower|).
  \item \verb|to_upper(s: string): string|\\
    Returns a copy of \verb|s| with the lowercase letters (as indicated by
    \verb|isascii| and \verb|islower|) folded to lowercase (via
    \verb|toupper|).
  \item \verb|is_ascii(s: string): bool|\\
    Returns false if any byte value of \texttt{s} is greater than 127, and true
    otherwise.
  \item \verb|edit(s: string, edit_char: string): string|\\
    Returns a version of \verb|s| assuming that \verb|edit_char| is the
    ``backspace character'' (usually \verb|\x08| for backspace or \verb|\x7f|
    for DEL). For example, \verb|edit("hello there", "e")| returns
    \verb|"llo t"|. The argument \verb|edit_char| must be a string of exactly
    one character, or Bro generates a run-time error and uses the first
    character in the string.
  \item \verb|clean(s: string): string|\\
    Replaces non-printable characters in \texttt{s} with escaped sequences,
    with the mappings
    \verb|NUL| $\rightarrow$ \verb|\0|,
    \verb|DEL| $\rightarrow$ \verb|^?|,
    values $\le 26$ $\rightarrow$ \verb|^[A-Z]|,
    and values not in $[32, 126]$~$\rightarrow$~\verb|%XX|. If the string does
    not yet have a trailing NUL, one is added.
  \item \verb|to_string_literal(s: string): string|\\
    Same as clean, but with different mappings:
    values not in $[32, 126]$~$\rightarrow$~\verb|%XX|,
    \verb|\|~$\rightarrow$~\verb|\\|,
    \verb|'|~$\rightarrow$~\verb|\'|,
    \verb|"|~$\rightarrow$~\verb|\"|.
  \item \verb|escape_string(s: string): string|\\
    Returns a printable version of \texttt{s}. Same as \texttt{clean} except
    that non-printable characters are removed.
  \item \verb|string_to_ascii_hex(s: string): string|\\
    Returns an ASCII hexadecimal representation of a string.
  \item \verb|strip(s: string): string|\\
    Strips whitespace at both ends of \texttt{s}.
  \item \verb|string_fill(len: int, source: string): string|\\
    Generates a string of size \texttt{len} and fills it with repetitions of
    \texttt{source}.
  \item \verb|str_shell_escape(source: string): string|\\
    Takes a string and escapes characters that would allow execution of
    commands at the shell level. Must be used before including strings in
    \verb|system| or similar calls.
  \item \verb|find_all(s: string, re: pattern): set of string|\\
    Returns all occurrences of \texttt{re} in \texttt{s} (or an empty empty set
    if none).
  \item \verb|find_last(s: string, re: pattern): string|\\
    Returns the last occurrence of \texttt{re} in \texttt{s}. If not found,
    returns an empty string.  Note that this function returns the match that
    starts at the largest index in the string, which is not necessarily the
    longest match.  For example, a pattern of \texttt{/.*/} will return the
    final character in the string.
  \item \verb|hexdump(data: string): string|\\
    Returns a hex dump for \texttt{data}. The hex dump renders 16 bytes per
    line, with hex on the left and ASCII (where printable) on the right. Based
    on Netdude's hex editor code.
  \item \verb|find_entropy(data: string): entropy_test_result|\\
    Performs an \href{http://www.fourmilab.ch/random/}{entropy}
    test on \verb|data|.
\verbose{
    The result is a record with the following fields:
    \begin{itemize}
      \item \verb|entropy|: The information density expressed as a number of
        bits per character.
      \item \verb|chi_square|: The $\chi^2$ test value expressed as an absolute
        number and a percentage which indicates how frequently a truly random
        sequence would exceed the value calculated, i.e., the degree to which
        the sequence tested is suspected of being non-random.
        If the percentage is greater than 99\% or less than 1\%, the sequence
        is almost certainly not random. If the percentage is between 99\% and
        95\% or between 1\% and 5\%, the sequence is suspect. Percentages
        between 90\% and 95\% and 5\% and 10\% indicate the sequence is
        ``almost suspect.''
      \item \verb|mean|: The arithmetic mean of all the bytes. If the data are
        close to random, it should be around 127.5.
      \item \verb|monte_carlo_pi|: Each successive sequence of six bytes is
        used as 24-bit $x$ and $y$ coordinates within a square. If the
        distance of the randomly-generated point is less than the radius of a
        circle inscribed within the square, the six-byte sequence is considered
        a ``hit.'' The percentage of hits can be used to calculate the value of
        $\pi$. For very large streams the value will approach the correct value
        of $\pi$ if the sequence is close to random.
      \item \verb|serial_correlation|: This quantity measures the extent to
        which each byte in the file depends upon the previous byte. For random
        sequences this value will be close to zero.
    \end{itemize}
}
  \item \verb|entropy_test_init(index: any): bool|\\
    Initializes data structures for incremental entropy calculation. The
    \verb|index| argument is an arbitrary unique value per distinct
    computation.
    \ReturnsTrueOnSuccess
    See \verb|entropy_test_add| and \verb|entropy_test_finish|.
  \item \verb|entropy_test_add(index: any, data: string): bool|\\
    Adds \verb|data| to the incremental entropy calculation identified by
    \verb|index|.
    \ReturnsTrueOnSuccess
  \item \verb|entropy_test_finish(index: any): entropy_test_result|\\
    Finalizes the incremental entropy calculation identified by \verb|index|.
    When all data has been added, this function returns the result record which
    is described above in \verb|find_entropy|.
\end{itemize}

\subsubsection*{Network Type Processing}

\begin{itemize}
  \item \verb|mask_addr(a: addr, top_bits_to_keep: count): subnet|\\
    Returns the address \verb|a| masked down to the number of upper bits
    indicated by \verb|top_bits_to_keep|, which must be greater than 0 and less
    than 33. For example, \verb|mask_addr(1.2.3.4, 18)| returns \verb|1.2.0.0|,
    and \verb|mask_addr(1.2.255.4, 18)| returns \verb|1.2.192.0|.
  \item \verb|remask_addr(a1: addr, a2: addr, top_bits_from_a1: count): count|\\
    Takes some top bits (e.g., subnet address) from \texttt{a1} and the other
    bits (intra-subnet part) from \texttt{a2} and merges them to get a new
    address. This is useful for anonymizing at subnet level while preserving
    serial scans.
  \item \verb|is_tcp_port(p: port): bool|\\
    Checks whether \texttt{p} is a TCP port.
  \item \verb|is_udp_port(p: port): bool|\\
    Checks whether \texttt{p} is a UDP port.
  \item \verb|is_icmp_port(p: port): bool|\\
    Checks whether \texttt{p} is an ICMP port.
  \item \verb|connection_exists(id: conn_id): bool|\\
    Checks whether the connection identified by \texttt{id} is (still) active.
  \item \verb|lookup_connection(id: conn_id): connection|\\
    Returns the \texttt{connection} record for \texttt{id}. If
    \texttt{id} does not point to an existing connection, the function
    generates a run-time error and returns a dummy value.
\verbose{
  \item \verb|get_conn_transport_proto(id: conn_id): transport_proto|\\
    Returns the transport protocol of the connection identified by \texttt{id}.
  \item \verb|get_port_transport_proto(p: port): transport_proto|\\
    Returns the transport protocol of \texttt{p}.
  \item \verb|get_orig_seq(id: conn_id): count|\\
    Returns the highest sequence number sent by a connection's originator, or 0
    if \verb|id| does not point to an active TCP connection. Sequence numbers
    are absolute (i.e., they reflect the values seen directly in packet
    headers; they are not relative to the beginning of the connection).
  \item \verb|get_resp_seq(id: conn_id): count|\\
    Returns the highest sequence number sent by a connection's responder, or 0
    if \verb|id| does not point to an active TCP connection.
}
  \item \verb|unescape_URI(URI: string): string|\\
    Unescapes all characters in \texttt{URI}, i.e., decodes every \verb|%xx|
    group.
% TODO: wait for answer on bro-dev about these functions.
%  \item \verb|preserve_prefix(a: addr, width: count)|\\
%  \item \verb|preserve_subnet(sn: subnet)|\\
%  \item \verb|anonymize_addr(a: addr, cl: IPAddrAnonymizationClass): addr|\\
  \item \verb|lookup_location(a: addr) : geo_location|\\
    Performs a geo-lookup of the IP address \verb|a|. Returns country, region,
    city, latitude, and longitude. Needs Bro to built with \texttt{libgeoip}.
  \item \verb|lookup_asn(a: addr): count|\\
    Performs an AS lookup of the IP address \verb|a|.
    Needs \texttt{libgeoip}.
  \item
\begin{verbatim}
x509_verify(der_cert: string, cert_stack: vector of string,
            root_certs: table[string] of string): count
\end{verbatim}
    Verifies the X.509 certificate in DER format given by \verb|der_cert|. The
    argument \verb|cert_stack| specifies a certificate chain to validate
    against, with index 0 typically being the root CA. Bro uses the Mozilla
    root CA list by default; \verb|root_certs| extends that list with
    additional root certificates.
  \item \verb|x509_err2str(err_num: count): string|\\
    Converts the X.509 certificate verification error code \verb|err_num| into
    a string representation.
\end{itemize}

\subsubsection*{Conversion}

\begin{itemize}
  \item \verb|cat(...): string|\\
    Returns the concatenation of the string representation of its arguments,
    which can be of any type. For example, \verb|cat("foo", 3, T)| returns
    \verb|"foo3T"|.
  \item \verb|cat_sep(sep: string, default: string, ...): string|\\
    Similar to cat, but places \texttt{sep} between each given argument.
    If any of the variable arguments is an empty string it is replaced by
    \texttt{default} instead.
  \item \verb|fmt(...): string|\\
    Produces a formatted string à la \verb|printf|.
\verbose{
    The first argument is the \emph{format string} and specifies how subsequent
    arguments are converted for output. It is composed of zero or more
    directives: ordinary characters (not \verb|%|), which are copied unchanged
    to the output, and conversion specifications, each of which fetches zero or
    more subsequent arguments. Conversion specifications begin with \verb|%|
    and the arguments must properly correspond to the specifier.
    After the \verb|%|, the following characters may appear in sequence:
    \begin{tabular}{l l}
      \verb|%| & Literal \verb|%|\\
      \verb|-| & Left-align field\\
      \verb|[0-9]+| & The field width (< 128)\\
      \verb|.| & Precision of floating point specifiers \verb|[efg]| (< 128)\\
      \verb|A| & Escape NUL bytes, i.e., replace \verb|0| with \verb|\0|\\
      \verb|[DTdxsefg]| &
        \begin{minipage}[t]{\linewidth}
          Format specifier\\
          \begin{tabular}{l p{.7\linewidth}}
            \texttt{[DT]} & ISO timestamp with microsecond precision\\
            \texttt{d} & Signed/Unsigned integer (using C-style
              \verb|%lld|/\verb|%llu| for \texttt{int}/\texttt{count})\\
            \texttt{x} & Unsigned hexadecimal (using C-style \verb|%llx|);
              addresses/ports are converted to host-byte order\\
            \texttt{s} & Escaped string\\
            \texttt{[efg]} & Double\\
          \end{tabular}
        \end{minipage}\\
    \end{tabular}
}
    Given no arguments, \verb|fmt| returns an empty string. Given a non-string
    first argument, \verb|fmt| returns the concatenation of all its arguments,
    per \verb|cat|. Finally, given the wrong number of additional arguments for
    the given format specifier, \verb|fmt| generates a run-time error.
  \item \verb|to_int(s: string): int|\\
    Converts a \texttt{string} into a (signed) integer.
  \item \verb|int_to_count(n: int): count|\\
    Converts a positive integer into a \texttt{count} or returns 0 if
    \texttt{n < 0}.
  \item \verb|double_to_count(d: double): count|\\
    Converts a positive \texttt{double} into a \texttt{count} or returns 0 if
    \texttt{d < 0.0}.
  \item \verb|to_count(s: string): count|\\
    Converts a \texttt{string} into a \texttt{count}.
  \item \verb|interval_to_double(i: interval): double|\\
    Converts an \texttt{interval} time span into a \texttt{double}.
  \item \verb|double_to_interval(d: double): interval|\\
    Converts a \texttt{double} into an \texttt{interval}.
  \item \verb|time_to_double(t: time): double|\\
    Converts a \texttt{time} value into a \texttt{double}.
  \item \verb|double_to_time(d: double): time|\\
    Converts a \texttt{double} into a \texttt{time} value.
  \item \verb|double_to_time(d: double): time|\\
    Converts a \texttt{double} into a \texttt{time} value.
  \item \verb|port_to_count(p: port): count|\\
    Returns the port number of \texttt{p} as \texttt{count}.
  \item \verb|count_to_port(num: count, t: transport_proto): port|\\
    Creates a \texttt{port} with number \texttt{num} and transport protocol
    \texttt{t}.
  \item \verb|to_port(s: string): port|\\
    Converts a \texttt{string} into a \texttt{port}.
  \item \verb|addr_to_count(a: addr): count|\\
    Converts an IP address into a 32-bit unsigned integer.
  \item \verb|count_to_v4_addr(ip: count): addr|\\
    Converts an unsigned integer into an IP address.
  \item \verb|to_addr(ip: string): addr|\\
    Converts a \texttt{string} into an IP address.
  \item \verb|raw_bytes_to_v4_addr(b: string): addr|\\
    Converts a \texttt{string} of bytes into an IP address. It interprets the
    first 4 bytes of \texttt{b} as an IPv4 address in network order.
  \item \verb|ptr_name_to_addr(s: string): addr|\\
    Converts a reverse pointer name to an address, e.g.,
    \verb|1.0.168.192.in-addr.arpa| to \verb|192.168.0.1|.
  \item \verb|addr_to_ptr_name(a: addr): string|\\
    Converts an IP address to a reverse pointer name, e.g.,
    \verb|192.168.0.1| to \verb|1.0.168.192.in-addr.arpa|.
  \item \verb|parse_dotted_addr(s: string): addr|\\
    Converts a decimal dotted IP address in a string to an address type.
\verbose{
  \item \verb|parse_ftp_port(s: string): ftp_port|\\
    Converts a string representation of the FTP PORT command to an
    \verb|ftp_port|,
    e.g., \verb|"10,0,0,1,4,31"| to \verb|[h=10.0.0.1, p=1055/tcp, valid=T]|
  \item \verb|parse_eftp_port(s: string): ftp_port|\\
    Same as as \verb|parse_ftp_port|, but instead for EPRT
    (see~\href{http://tools.ietf.org/html/rfc2428}{RFC 2428}) whose format is
    \verb|EPRT<space><d><net-prt><d><net-addr><d><tcp-port><d>|, where
    \verb|<d>| is a delimiter in the ASCII range 33-126 (usually \verb#|#).
  \item \verb|parse_ftp_pasv(s: string): ftp_port|\\
    Converts the result of the FTP PASV command to an \verb|ftp_port|.
  \item \verb|parse_ftp_epsv(s: string): ftp_port|\\
    Same as \verb|parse_ftp_pasv|, but instead for the EPSV
    (see~\href{http://tools.ietf.org/html/rfc2428}{RFC 2428}) whose format is
    \verb|<text> (<d><d><d><tcp-port><d>)|, where \verb|<d>| is a delimiter in
    the ASCII range 33-126 (usually \verb#|#).
  \item \verb|fmt_ftp_port(a: addr, p: port): string|\\
    Formats the IP address \texttt{a} and TCP port \texttt{p} as an FTP
    PORT command, e.g., \verb|10.0.0.1| and \verb|1055/tcp| to
    \verb|"10,0,0,1,4,31"|.
  \item \verb|decode_netbios_name(name: string): string|\\
    Decodes a \href{http://support.microsoft.com/kb/194203}{NetBIOS name}, e.g.,
    \verb|"FEEIEFCAEOEFFEECEJEPFDCAEOEBENEF"| to \verb|"THE NETBIOS NAME"|.
  \item \verb|decode_netbios_name_type(name: string): count|\\
    Converts the \href{http://support.microsoft.com/kb/163409}{NetBIOS name
    type} to the corresponding numeric value.
}
  \item \verb|bytestring_to_hexstr(bytestring: string): string|\\
    Converts a string of bytes into its hexadecimal representation, e.g.,
    \verb|"04"| to \verb|"3034"|.
  \item \verb|decode_base64(s: string): string|\\
    Decodes the Base64-encoded string \verb|s|.
  \item \verb|decode_base64_custom(s: string, a: string): string|\\
    Decodes the Base64-encoded string \verb|s| with alphabet \verb|a|.
  \item \verb|uuid_to_string(uuid: string): string|\\
    Converts a bytes representation of a
    \href{http://en.wikipedia.org/wiki/Universally_unique_identifier}{UUID} to
    its string form, e.g., to \verb|550e8400-e29b-41d4-a716-446655440000|.
  \item \verb|merge_pattern(p1: pattern, p2: pattern): pattern|\\
    Merges and compiles the regular expressions \verb|p1| and \verb|p2| at
    initialization time (e.g., in the event \verb|bro_init()|).
  \item \verb|convert_for_pattern(s: string): string|\\
    Escapes \verb|s| so that it is a valid pattern and can be used with the
    \verb|string_to_pattern|. Concretly, any character
    from the set \verb#^$-:"\/|*+?.(){}[]# is prefixed with \verb|\|.
  \item \verb|string_to_pattern(s: string, convert: bool): pattern|\\
    Converts \verb|s| into a pattern. If \verb|convert| is true, \verb|s| is
    first passed through the function \verb|convert_for_pattern| to escape
    special characters of patterns.
\verbose{
  \item \verb|mode2string(mode: count): string|\\
    Converts UNIX file permissions given by \verb|mode| to a string
    representation of the form \verb|rw[xsS]rw[xsS]rw[xtT]|.
}
\end{itemize}

\end{multicols*}

\end{document}
